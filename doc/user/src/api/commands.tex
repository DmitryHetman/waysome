\subsection{Commands}

A command is a single operation inside a transaction. We call it ``command''
because it can only be a single command, combinations are not possible.

A list of all available commands follows.

\subsubsection{Arithmetical commands}

    \begin{commands}

        \command{add}
                {N : Numerical}
                {Numerical}
                {
                    Adds values passed via a stack position or directly and
                    pushes the result to the stack.
                }

        \command{sub}
                {N : Numerical}
                {Numerical}
                {
                    Subtracts values passed via a stack position or directly
                    and pushes the result to the stack.
                }

        \command{mul}
                {N : Numerical}
                {Numerical}
                {
                    Multiplicates values passed via a stack position or directly
                    and pushes the result to the stack.
                }

        \command{div}
                {Numerical, Numerical}
                {Numerical}
                {
                    Divides first argument by second argument.

                    Fails when the second argument (divisor) is zero.
                }

    \end{commands}

\subsubsection{Logical commands}

    \begin{commands}

        \command{lnot}
                {Value}
                {Boolean}
                {
                    Performs a logical NOT on the argument, whereas only a NIL
                    and a Boolean which is false are actually false and produce
                    a return value which is true.
                }

        \command{land}
                {N : Value}
                {Boolean}
                {
                    Logical AND between all arguments, whereas only NIL and a
                    Boolean which is false evaluate to false.
                }

        \command{lnand}
                {N : Value}
                {Boolean}
                {
                    Logical NAND between all arguments, whereas only NIL and a
                    Boolean which is false evaluate to false.
                }

        \command{lor}
                {N : Value}
                {Boolean}
                {
                    Logical OR between all arguments, whereas only NIL and a
                    Boolean which is false evaluate to false.
                }

        \command{lnor}
                {N : Value}
                {Boolean}
                {
                    Logical NOR between all arguments, whereas only NIL and a
                    Boolean which is false evaluate to false.
                }

        \command{lxor}
                {N : Value}
                {Boolean}
                {
                    Logical XOR between all arguments, whereas only NIL and a
                    Boolean which is false evaluate to false.
                }

    \end{commands}

\subsubsection{Bitwise commands}

    \begin{commands}

        \command{bnot}
                {Integer}
                {Boolean}
                {
                    Does a logical NOT on the integer.
                }

        \command{band}
                {N : Integer}
                {Boolean}
                {
                    Does a logical AND between all arguments.
                }

        \command{bnand}
                {N : Integer}
                {Boolean}
                {
                    Does a bitwise AND between all arguments and does a binary
                    NOT on that value.
                }

        \command{bor}
                {N : Integer}
                {Boolean}
                {
                    Does a binary OR between all arguments.
                }

        \command{bnor}
                {N : Integer}
                {Boolean}
                {
                    Does a binary OR between all arguments and does a binary
                    NOT on that value.
                }

        \command{bxor}
                {N : Integer}
                {Boolean}
                {
                    Does a binary XOR between all arguments.
                }

    \end{commands}

% Special commands

    % \begin{commands}
    %
    % \end{commands}
